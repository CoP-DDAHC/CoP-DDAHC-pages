\documentclass[a4paper,12pt]{article}
\usepackage[utf8]{inputenc}
\usepackage[slovak]{babel}
\usepackage{enumitem}
\usepackage{graphicx}
\usepackage{hyperref}
\usepackage{titlesec}
\usepackage{xcolor}

% Nastavenie farby odkazov
\hypersetup{
    colorlinks=true,
    linkcolor=blue,
    filecolor=magenta,
    urlcolor=blue,
    pdftitle={Zápisnica - Predspracovanie obrázkov},
    pdfpagemode=FullScreen,
}

% Definovanie farieb pre sekcie
\titleformat{\section}
  {\normalfont\Large\bfseries\color{blue}}{}{0em}{}
\titleformat{\subsection}
  {\normalfont\large\bfseries\color{blue}}{}{0em}{}
\titleformat{\subsubsection}
  {\normalfont\normalsize\bfseries\color{blue}}{}{0em}{}

\begin{document}

\subsubsection{Predspracovanie obrázkov – 31. októbra 2025}

\textbf{Dátum:} 31. októbra 2025 \\
\textbf{Čas:} 14:00 – 14:45 \\
\textbf{Umiestnenie:} Online – Discord \\
\textbf{Zapisovateľ:} Bc. Zoltán Renczes

\vspace{0.5cm}
\textbf{Účastníci:}
\begin{itemize}
    \item Bc. Michal Balogh – ML/AI Engineer
    \item Bc. Juraj Hušek – Backend Developer
    \item Bc. Ján Osadský – Backend Developer
    \item Bc. Fridrich Molnár – Frontend \& Server Developer
    \item Bc. Zoltán Renczes – Frontend Developer
\end{itemize}

\textbf{Vedúci tímového projektu:} Ing. Stanislav Marochok

\vspace{0.5cm}
\textbf{Program stretnutia:}
\begin{enumerate}
    \item Predspracovanie datasetu pre YOLO
    \item Predspracovanie obrazových dát
    \item Overenie e-mailu a validácia (backend)
    \item Aktualizácia frontendu
\end{enumerate}

\vspace{0.5cm}
\textbf{Diskusia a rozhodnutia:}

\paragraph{1. Prezentovanie predspracovania datasetu pre YOLO}
Vedúci projektu odprezentoval skript \texttt{preprocess\_yolo\_datatset.py}, ktorý slúži na automatické predspracovanie datasetu pre trénovanie modelu YOLO. Skript upravuje kvalitu obrázkov a zachováva YOLO anotácie bounding boxov. Výsledkom je nová verzia datasetu v rovnakej štruktúre (train/val/test), pripravená na efektívnejšie trénovanie modelu.

\paragraph{2. Predspracovanie obrázkových dát v projekte}
Začala implementácia predspracovania obrazových dát. Práca je v počiatočnej fáze, cieľom je zabezpečiť správne spracovanie vstupných obrazov.

\paragraph{3. Overenie e-mailu a validácia vstupov na strane backendu}
Do backendu bola pridaná funkcionalita overenia e-mailovej adresy pri registrácii používateľov (okrem Google OAuth). Implementácia zahŕňa generovanie bezpečného JWT tokenu, odosielanie overovacieho e-mailu cez SMTP (s využitím premenných v súbore \texttt{.env}) a spracovanie verifikácie na koncovom bode \texttt{/verify-email/\{token\}} s presmerovaním na frontend. Po úspešnom overení sa používateľ označí ako overený.

Doplnená bola backendová validácia vstupných údajov pri registrácii, zosúladená s kritériami validácie na frontende. Aktualizovaná bola konfigurácia súboru \texttt{.env} podľa usmernení v dokumente \textit{ENV\_GUIDE} a referenčného súboru \texttt{.env\_example}.

\paragraph{4. Aktualizácie frontend časti aplikácie}
Do projektu bola pridaná komponenta \texttt{Home.jsx} a aktualizované viaceré frontendové komponenty na zlepšenie funkcionality prihlasovania a registrácie. Implementované boli toast notifikácie pre systémové správy a zobrazovanie nahraných súborov, ktoré sú dostupné na prehliadnutie priamo z používateľského rozhrania. Doplnená bola implementácia routera pre navigáciu medzi stránkami aplikácie.

\vspace{0.5cm}
\textbf{Úlohy do nasledujúceho stretnutia:}
\begin{itemize}
    \item Michal Balogh: Predspracovanie nahratých dokumentov
    \item Bc. Juraj Hušek: Implementovať uloženie obrázkov v databáze na strane backendu
    \item Bc. Ján Osadský: Implementovať zobrazenie, načítanie a správu uložených obrázkov
    \item Bc. Fridrich Molnár: Implementovať vylepšenia anotačného nástroja a integrovať serverovú časť
    \item Bc. Zoltán Renczes: Implementovať vizuálne vylepšenia UI (zmena kurzora, zoom ku kurzoru) a optimalizácia UI/UX anotačného nástroja
\end{itemize}

\vspace{0.5cm}
\textbf{Nasledujúce stretnutie:} \\
Dátum: 6. novembra 2025 \\
Čas: 12:00 \\
Umiestnenie: Unispace

\end{document}
