\documentclass[a4paper,12pt]{article}
\usepackage[utf8]{inputenc}
\usepackage[slovak]{babel}
\usepackage{enumitem}
\usepackage{graphicx}
\usepackage{fancyhdr}
\usepackage{geometry}
\geometry{top=2.5cm,bottom=2.5cm,left=2.5cm,right=2.5cm}

\title{Zápisnica zo stretnutia}
\author{}
\date{}

\begin{document}

\maketitle

\subsubsection{Stretnutie ohľadom Docker setupu a klasifikácie dokumentov – 9. októbra 2025}

\textbf{Dátum:} 9. októbra 2025 \\
\textbf{Čas:} 12:00 – 12:20 \\
\textbf{Umiestnenie:} Slovenská technická univerzita v Bratislave, Fakulta elektrotechniky a informatiky \\
\textbf{Miestnosť:} A208, FEI STU \\
\textbf{Zapisovateľ:} Bc. Fridrich Molnár

\vspace{0.5cm}
\textbf{Účastníci:}
\begin{itemize}
    \item Bc. Michal Balogh – ML/AI Engineer
    \item Bc. Juraj Hušek – Backend Developer (ospravedlnený)
    \item Bc. Ján Osadský – Backend Developer (ospravedlnený)
    \item Bc. Fridrich Molnár – Frontend \& Server Developer
    \item Bc. Zoltán Renczes – Frontend Developer
\end{itemize}

\textbf{Vedúci tímového projektu:} Ing. Stanislav Marochok

\vspace{0.5cm}
\textbf{Program stretnutia:}
\begin{enumerate}
    \item Prehľad Docker setupu
    \item Diskusia o klasifikácii dokumentov
    \item Úlohy do nasledujúceho stretnutia
\end{enumerate}

\vspace{0.5cm}
\textbf{Diskusia a rozhodnutia:}

\paragraph{1. Docker setup}
Účastníci zhodnotili aktuálny stav kontajnerov (backend, databáza, frontend, AI service). Všetky kontajnery sú prepojené a komunikujú správne. Diskutovalo sa o stabilite prostredia a možnostiach ďalšieho rozvoja serverovej časti. Zhodli sme sa, že setup je pripravený pre integráciu klasifikácie a frontend funkcionality.

\paragraph{2. Klasifikácia dokumentov}
Prebehla demonštrácia OCR a klasifikácie dokumentov. Systém dokáže rozpoznať dokument a zamietnuť všetko, čo nie je dokument. Diskutovalo sa o možnom nasadení tejto funkcionality do aplikácie a o spôsobe ukladania logov a výsledkov do databázy. Zhodli sme sa, že klasifikáciu je možné integrovať do aplikácie v najbližšom kroku.

\paragraph{3. Plán nasledujúcich krokov}
Prebehla diskusia o prioritách ďalšieho vývoja: integrácia klasifikácie, dokončenie frontendu podľa možností, návrh backend riešení, spustenie servera a sprístupnenie databázy pre ukladanie dokumentov a logov. Zhodli sme sa, že je potrebné pripraviť registráciu a prihlásenie používateľov a rozdeliť úlohy medzi členov tímu.

\paragraph{Úlohy do nasledujúceho stretnutia}
\begin{itemize}
    \item Bc. Michal Balogh: Pokračovať v práci na klasifikácii a AI nástrojoch
    \item Bc. Juraj Hušek: Navrhnúť backend riešenie a sprístupniť databázu
    \item Bc. Ján Osadský: Implementácia backendovej funkcionality pre registráciu a prihlásenie
    \item Bc. Fridrich Molnár: Dokončiť frontend – implementovať registráciu a prihlásenie a integrovať serverovú časť
    \item Bc. Zoltán Renczes: Dolaďovanie vizuálneho rozloženia frontendu, optimalizácia UI/UX
\end{itemize}

\vspace{0.5cm}
\textbf{Nasledujúce stretnutie:} \\
Dátum: 23. októbra 2025 \\
Čas: 16:00 \\
Umiestnenie: Online

\vspace{0.5cm}
\textbf{Poznámky:}
\begin{itemize}
    \item Tím sa zhodol, že Docker setup a environment fungujú stabilne a sú pripravené na integráciu klasifikácie a frontendu.
    \item OCR a klasifikácia dokumentov fungujú správne.
    \item Tím sa dohodol na pravidelných konzultáciách a koordinácii pri integrácii frontendu a backendu, aby sa predišlo problémom a zabezpečila sa hladká implementácia ďalších funkcionalít.
\end{itemize}

\end{document}
