\documentclass[12pt,a4paper]{article}
\usepackage[utf8]{inputenc}
\usepackage[slovak]{babel}
\usepackage{hyperref}
\usepackage{enumitem}
\usepackage{geometry}
\usepackage{graphicx}
\graphicspath{{doc/img/}}
\geometry{margin=2.5cm}


\title{Implementácia Backendu}
\author{}
\date{}
\begin{document}
\maketitle
Backendová časť systému bola implementovaná ako samostatná webová služba,
ktorá zabezpečuje autentifikáciu používateľov, správu dokumentov, komunikáciu
s AI modulmi a prácu s databázou. Backend vystupuje ako centrálny integračný
prvok celého systému a sprostredkúva komunikáciu medzi aplikačnou vrstvou,
dátovým úložiskom a externým AI modulom.

\section{Použité technológie}

Backend systému bol implementovaný v programovacom jazyku Python s využitím
frameworku \textbf{FastAPI}~\cite{fastapi}. Tento framework sme zvolili najmä
pre jeho vysoký výkon, natívnu podporu asynchrónneho spracovania požiadaviek a
automatické generovanie OpenAPI dokumentácie.

Na prácu s databázou sme použili ORM nástroj \textbf{SQLAlchemy}~\cite{sqlalchemy},
ktorý umožňuje objektovo-orientovaný prístup k relačnej databáze a zjednodušuje
správu dátových modelov. Ako databázový systém je použitá
\textbf{PostgreSQL}~\cite{postgresql}, ktorá poskytuje spoľahlivé ukladanie dát
a podporu pre komplexnejšie dátové štruktúry.

Autentifikácia používateľov je realizovaná kombináciou
\textbf{JSON Web Tokens (JWT)}~\cite{jwt} a \textbf{OAuth~2.0}~\cite{oauth2}.
Tento prístup umožňuje podporu klasickej registrácie používateľov, ako aj
prihlásenie prostredníctvom externého poskytovateľa identity
(Google OAuth).

Komunikácia s AI modulmi prebieha prostredníctvom REST API s využitím knižnice
\textbf{httpx}~\cite{httpx}, ktorá umožňuje asynchrónne odosielanie požiadaviek
a spracovanie odpovedí. Backend je pripravený na beh v kontajnerizovanom
prostredí \textbf{Docker}~\cite{docker}, čo zjednodušuje nasadenie a integráciu
jednotlivých komponentov systému.

\section{Architektúra backendu}

Backend je navrhnutý ako modulárna aplikácia, v ktorej sú jednotlivé časti
systému oddelené podľa ich zodpovednosti. Základ aplikácie tvorí hlavný
vstupný bod, ktorý inicializuje FastAPI aplikáciu, konfiguruje middleware
a registruje jednotlivé API endpointy.

Architektúra backendu zahŕňa samostatné moduly pre:
\begin{itemize}
    \item správu používateľov a autentifikáciu,
    \item definíciu dátových modelov a databázových entít,
    \item validáciu vstupných a výstupných dát pomocou schém,
    \item bezpečnostné mechanizmy (hashovanie hesiel, overovanie tokenov),
    \item pomocné nástroje a utility funkcie.
\end{itemize}

Backend využíva dependency injection mechanizmus frameworku FastAPI,
ktorý umožňuje efektívne spravovať databázové spojenia a autentifikačný
kontext používateľa. Prístup k databáze je realizovaný prostredníctvom
databázovej session, ktorá je automaticky sprístupnená jednotlivým endpointom.

Dôležitou súčasťou architektúry je oddelenie backendu od AI modulov.
AI spracovanie dokumentov prebieha v samostatnej službe, ku ktorej backend
pristupuje prostredníctvom REST API. Tento prístup umožňuje nezávislý vývoj
a škálovanie AI časti systému bez zásahu do backendovej logiky.

\section{Autentifikácia a autorizácia}

Autentifikácia a autorizácia používateľov sú v systéme implementované ako
kľúčové bezpečnostné mechanizmy, ktoré zabezpečujú kontrolovaný prístup
k funkcionalitám backendu. Backend podporuje kombináciu klasickej autentifikácie
pomocou používateľského mena a hesla a autentifikáciu prostredníctvom externého
poskytovateľa identity založenú na štandarde OAuth~2.0.

Pri klasickej registrácii používateľa je heslo pred uložením do databázy
hashované pomocou kryptografickej hashovacej funkcie. Po úspešnom prihlásení
je používateľovi vygenerovaný JSON Web Token (JWT), ktorý obsahuje identifikátor
používateľa a ďalšie nevyhnutné informácie. Tento token je následne využívaný
pri autorizácii jednotlivých API požiadaviek.

Overovanie JWT tokenu prebieha pomocou dependency injection
mechanizmu frameworku FastAPI. Chránené endpointy vyžadujú platný token,
ktorý je kontrolovaný z hľadiska platnosti, integrity a časovej expirácie.
Na základe dekódovaného tokenu je identifikovaný aktuálne prihlásený používateľ.

Súčasťou autentifikačného procesu je aj verifikácia e-mailovej adresy
používateľa. Po registrácii je používateľovi zaslaný overovací e-mail
obsahujúci jednorazový token. Až po úspešnom overení e-mailovej adresy
je používateľovi povolený prístup k chráneným funkcionalitám systému.

Autentifikácia prostredníctvom OAuth~2.0 je realizovaná s využitím
externého poskytovateľa identity \textbf{Google}. OAuth konfigurácia
je spravovaná v prostredí \textbf{Google Cloud Platform}~\cite{gcp_oauth},
kde je registrovaná aplikácia systému a definované oprávnenia na prístup
k základným informáciám o používateľovi.

Autorizácia je založená na kontrole identity používateľa a jeho oprávnení
pri jednotlivých operáciách. Backend overuje, či má používateľ právo
vykonávať požadované operácie.

\subsection{Integrácia AI modulov}

Integrácia AI modulov je v systéme realizovaná prostredníctvom samostatnej
služby, ktorá je oddelená od backendovej aplikácie. Backend zabezpečuje komunikáciu medzi
aplikačnou vrstvou a AI modulmi.

Komunikácia s AI modulmi prebieha prostredníctvom REST API. Backend odosiela
vstupné dáta, ako sú obrázky dokumentov alebo parametre spracovania, a
očakáva odpoveď obsahujúcu výsledky analýzy. Na realizáciu tejto komunikácie
je využitá asynchrónna HTTP knižnica, ktorá umožňuje efektívne spracovanie
požiadaviek bez blokovania hlavného aplikačného vlákna.

\subsection{Práca s databázou a dátový model}

Na ukladanie perzistentných dát systému je použitá relačná databáza
PostgreSQL. Prístup k databáze je realizovaný prostredníctvom ORM nástroja
SQLAlchemy, ktorý umožňuje mapovanie databázových tabuliek na objektové
reprezentácie v aplikačnej logike.

Dátový model systému je navrhnutý tak, aby podporoval správu používateľov,
historických dokumentov a výsledkov ich spracovania. Základnými entitami
sú používateľ, dokument a záznamy o zmenách vykonaných v systéme.
Dokumenty sú mapované na konkrétnych používateľov, čím je zabezpečené
vlastníctvo a kontrola prístupu k dátam.

Okrem samotných dokumentov databáza uchováva aj metadáta generované
AI modulmi, ako sú informácie o štruktúre dokumentu alebo výsledkoch
predspracovania. Súčasťou databázy sú aj záznamy o aktivitách používateľov,
ktoré umožňujú sledovanie zmien a audit operácií vykonaných v systéme.

Vzťahy medzi jednotlivými entitami a ich hlavné atribúty sú znázornené
v entitno-relačnom diagrame databázy, ktorý ilustruje štruktúru dátového
modelu a väzby medzi jeho jednotlivými časťami.

\centerline{\includegraphics[width=0.3\textwidth]{er_diagram.png}}


\subsection{API rozhranie}

Backend systému poskytuje REST API rozhranie, prostredníctvom ktorého
aplikačná vrstva komunikuje so systémom. API je navrhnuté v súlade so
štýlom REST a využíva štandardné HTTP metódy na realizáciu jednotlivých
operácií. Zahŕňa endpointy na správu používateľských účtov,
autentifikáciu, nahrávanie a správu dokumentov, ako aj spúšťanie
spracovania dokumentov prostredníctvom AI modulu. Prístup k vybraným
endpointom je chránený autentifikačnými mechanizmami a vyžaduje
platný JWT token.

FastAPI framework umožňuje automatické generovanie OpenAPI dokumentácie,
ktorá poskytuje prehľad dostupných endpointov, ich vstupných parametrov
a návratových hodnôt. Táto dokumentácia uľahčuje integráciu frontendu
a zjednodušuje testovanie backendových služieb.

API rozhranie je navrhnuté tak, aby podporovalo rozšíriteľnosť systému
a umožňovalo jednoduché dopĺňanie nových funkcionalít bez nutnosti
zásahu do existujúcich častí aplikácie.

\section*{Použité zdroje}
\begingroup
\renewcommand{\section}[2]{}%
\begin{thebibliography}{9}
\bibitem{fastapi}
Sebastián Ramírez,
\textit{FastAPI},
2024,
\url{https://fastapi.tiangolo.com/},
Dostupné z: \url{https://fastapi.tiangolo.com/} [cit. 2026-01-15]

\bibitem{sqlalchemy}
Mike Bayer,
\textit{SQLAlchemy},
2024,
\url{https://www.sqlalchemy.org/},
Dostupné z: \url{https://www.sqlalchemy.org/} [cit. 2026-01-15]

\bibitem{postgresql}
PostgreSQL Global Development Group,
\textit{PostgreSQL Documentation},
2024,
\url{https://www.postgresql.org/docs/},
Dostupné z: \url{https://www.postgresql.org/docs/} [cit. 2026-01-15]

\bibitem{jwt}
M. Jones, J. Bradley, N. Sakimura,
\textit{JSON Web Token (JWT)},
RFC 7519,
2015,
\url{https://datatracker.ietf.org/doc/html/rfc7519}

\bibitem{oauth2}
D. Hardt,
\textit{The OAuth 2.0 Authorization Framework},
RFC 6749,
2012,
\url{https://datatracker.ietf.org/doc/html/rfc6749}

\bibitem{httpx}
Encode OSS,
\textit{httpx: A next generation HTTP client for Python},
2024,
\url{https://www.python-httpx.org/},
Dostupné z: \url{https://www.python-httpx.org/} [cit. 2026-01-15]

\bibitem{docker}
Docker Inc.,
\textit{Docker Documentation},
2024,
\url{https://docs.docker.com/},
Dostupné z: \url{https://docs.docker.com/} [cit. 2026-01-15]

\bibitem{gcp_oauth}
Google LLC,
\textit{Google Identity Platform – OAuth 2.0},
2024,
\url{https://cloud.google.com/identity/docs/oauth2},
Dostupné z: \url{https://cloud.google.com/identity/docs/oauth2} [cit. 2026-01-15]

\end{thebibliography}
\endgroup

\end{document}