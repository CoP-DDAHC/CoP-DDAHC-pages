\documentclass[a4paper,12pt]{article}
\usepackage[utf8]{inputenc}
\usepackage[slovak]{babel}
\usepackage{enumitem}
\usepackage{graphicx}
\usepackage{fancyhdr}
\usepackage{lipsum}

\title{Zápisnica - Špecializácia požiadaviek a plánovanie}
\author{}
\date{}

\begin{document}

\maketitle

\subsubsection{Špecializácia požiadaviek a plánovanie – 2. októbra 2025}

\textbf{Dátum:} 2. októbra 2025 \\
\textbf{Čas:} 12:00 – 13:00 \\
\textbf{Umiestnenie:} Slovenská technická univerzita v Bratislave, Fakulta elektrotechniky a informatiky \\
\textbf{Miestnosť:} A208, FEI STU \\
\textbf{Zapisovateľ:} Bc. Ján Osadský

\vspace{0.5cm}
\textbf{Účastníci:}
\begin{itemize}
    \item Bc. Michal Balogh – ML/AI Engineer
    \item Bc. Juraj Hušek – Backend Developer
    \item Bc. Ján Osadský – Backend Developer
    \item Bc. Fridrich Molnár – Frontend Developer
    \item Bc. Zoltán Renczes – Frontend Developer (ospravedlnený)
\end{itemize}

\textbf{Vedúci tímového projektu:} Ing. Stanislav Marochok

\vspace{0.5cm}
\textbf{Program stretnutia:}
\begin{enumerate}
    \item Diskusia vývojárskeho prostredia a programovacích jazykov
    \item Špecializácia požiadaviek
    \item Rozšírenie funkcionality
    \item Stanovenie priorít dlhodobého plánovania
    \item Úlohy do nasledujúceho stretnutia
\end{enumerate}

\vspace{0.5cm}
\textbf{Diskusia a rozhodnutia:}

\paragraph{1. Prezentovanie progresu na projekte}
Účastníci si vopred pripravili materiály, ktorými vedúceho projektu oboznámili o najnovších postupoch a vykonaných úlohách. Medzi najdôležitejšie patrí: výber programovacieho prostredia a jazykov, data flow diagram, use case a diagram architektúry.

\paragraph{2. Špecializácia požiadaviek}
Na impulz vedúceho projektu prebehol detailný prechod požadovanou funkcionalitou, pričom prebehlo následné došpecifikovanie projektových požiadaviek. Vedúci kládol dôraz hlavne na možnosť upravovať dokument v každej fáze práce s ním a prebehla diskusia o live načítavaní upravovaného dokumentu. Následne sme diskutovali o bounding boxoch a segmentácií s využitím masky, pričom sme došli na konsenzus, že funkcionalita presnej masky je nie len nad rámec projektu ale aj nepraktická z pohľadu programu.

\paragraph{3. Rozšírenie funkcionality}
Vedúci projektu predstavil zopár dodatočných funkcionalít, ako medziužívateľské zdieľanie dokumentov a možnosť registrácie cez Google. Zároveň bolo spomenuté používanie ochrannej pečiatky na dokumentoch, aby sa právne predišlo zneužívaniu autorských práv. Okrem toho boli doriešené menšie nejasnosti ohľadom funkcionality administrátora.

\paragraph{4. Stanovenie priorít dlhodobého plánovania}
Absolútnou prioritou vedúceho projektu bolo v aktuálnej fáze vytvoriť proof of concept a MVP (minimum viable product).

\paragraph{5. Úlohy do nasledujúceho stretnutia}
\begin{itemize}
    \item Všetci členovia: detailne si preštudovať doplnenú funkcionalitu a špecifikácie projektu.
    \item Bc. Michal Balogh: štúdium použiteľných datasetov na trénovanie AI.
    \item Bc. Juraj Hušek, Bc. Ján Osadský: vytvorenie štruktúry programu v repozitári, aktualizovať use case diagramy a položiť architektúru projektu.
    \item Bc. Fridrich Molnár: Spustenie Raspberry Pi servera a začať proof of concept. Spustenie základnej funkcionality frontendu.
    \item Bc. Zoltán Renczes: Naštudovanie vhodného frontend prístupu a vizuálneho rozloženia. Optimalizácia prostredia pre zadanú funkcionalitu. Začať proof of concept.
\end{itemize}

\vspace{0.5cm}
\textbf{Nasledujúce stretnutie:} \\
Dátum: 9. októbra 2025 \\
Čas: 12:00 \\
Umiestnenie: Slovenská technická univerzita v Bratislave, Fakulta elektrotechniky a informatiky \\
Miestnosť: A208, FEI STU

\vspace{0.5cm}
\textbf{Poznámky:}
\begin{itemize}
    \item Tím sa zhodol, že niektoré funkcionality je správne doimplementovať až po spustení programu so základnou funkcionalitou.
    \item Zdôraznená bola potreba pravidelných konzultácií s vedúcim tímového projektu, ktoré umožnia včasnú identifikáciu a riešenie problémov a ďalších funkcionalít systému. Tím sa dohodol, že pravidelné hodinové štvrtkové stretnutia vyhovujú všetkým členom.
    \item Tím je stále v procese výberu vhodného mena pre aplikáciu.
\end{itemize}

\end{document}
