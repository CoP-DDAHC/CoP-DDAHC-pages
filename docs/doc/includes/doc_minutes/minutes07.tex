\documentclass[a4paper,12pt]{article}
\usepackage[utf8]{inputenc}
\usepackage[slovak]{babel}
\usepackage{enumitem}
\usepackage{graphicx}
\usepackage{fancyhdr}
\usepackage{lastpage}
\usepackage{hyperref}

% Nastavenie hlavičky a päty
\pagestyle{fancy}
\fancyhf{}
\fancyhead[L]{Zápisnica}
\fancyhead[R]{\today}
\fancyfoot[C]{\thepage\ / \pageref{LastPage}}

% Začiatok dokumentu
\begin{document}

\subsubsection{Rozšírenie datasetu a nové BE/FE funkcionality – 13. novembra 2025}

\textbf{Dátum:} 13. novembra 2025 \\
\textbf{Čas:} 12:00 – 13:00 \\
\textbf{Umiestnenie:} A208, FEI STU / Discord (online) \\
\textbf{Zapisovateľ:} Bc. Juraj Hušek

\vspace{0.5cm}
\textbf{Účastníci:}
\begin{itemize}
    \item Bc. Michal Balogh – ML/AI Engineer
    \item Bc. Juraj Hušek – Backend Developer (online)
    \item Bc. Ján Osadský – Backend Developer (online)
    \item Bc. Fridrich Molnár – Frontend \& Server Developer
    \item Bc. Zoltán Renczes – Frontend Developer
\end{itemize}

\textbf{Vedúci tímového projektu:} Ing. Stanislav Marochok

\vspace{0.5cm}
\textbf{Program stretnutia:}
\begin{enumerate}
    \item Odprezentovanie pokroku v implementácii
    \item Trénovanie na verejných datasetoch rukou písaného textu
    \item Segmentácia viet na slová
    \item Rozšírenie datasetu
\end{enumerate}

\vspace{0.5cm}
\textbf{Diskusia a rozhodnutia:}

\paragraph{1. Odprezentovanie pokroku v implementácii}

\textbf{Backend – nové funkcionality} \\
Bc. Juraj Hušek implementoval kompletné dokumentové API:
\begin{itemize}
    \item \texttt{GET /documents/user/\{user\_id\}} – zoznam dokumentov používateľa
    \item \texttt{GET /documents/\{document\_id\}} – načítanie dokumentu
    \item \texttt{DELETE /documents/\{document\_id\}} – bezpečné zmazanie dokumentu (iba vlastník)
    \item \texttt{GET /documents} – stránkovaný zoznam všetkých dokumentov
    \item \texttt{GET /documents/\{document\_id\}/download} – stiahnutie dokumentu
    \item všetky GET endpointy vracajú aj \textbf{thumbnail náhľady}
\end{itemize}
Nový endpoint \texttt{POST /preprocess/\{document\_id\}} posiela dokument na AI service na predspracovanie, výsledok sa vracia ako \texttt{StreamingResponse}, dostupné len pre autentifikovaných používateľov.

Opravená funkcionalita nahrávania dokumentov – dokumenty sa spoľahlivo ukladajú do databázy, generujú sa changelog záznamy a korektne sa ukladá informácia o vlastníkovi dokumentu.

\textbf{Frontend – vylepšenia} \\
Bc. Fridrich Molnár implementoval dragovateľné a resizovateľné bounding boxy v časti Upload, realtime vytváranie nového bounding boxu, toast notifikácie zobrazujúce výsledok akcií. Konfigurácia frontendu bola doplnená o upravené CORS nastavenia a aktualizované proxy URL.

\paragraph{2. Trénovanie na verejných datasetoch rukou písaného textu}
Bc. Michal Balogh informoval o postupe trénovania modelu AI service na veľkých verejných datasetoch rukou písaného textu. Cieľom je natrénovať základný model na množstve ručne písaných ukážok, následne finetuning na vlastnom datasete anotovaných historických dokumentov.

\paragraph{3. Segmentácia viet na slová}
Vedúci projektu Ing. Stanislav Marochok pokračoval vo výskume verejných datasetov s anotáciami na úrovni viet, ktoré spracováva na úroveň slov. Cieľom je porovnať sentence-level a word-level granularitu pri trénovaní.

\paragraph{4. Rozšírenie datasetu}
Diskutovalo sa o potrebe rozšírenia datasetu anotovaných historických dokumentov pre zvýšenie úspešnosti modelu.

\vspace{0.5cm}
\textbf{Prijaté rozhodnutia:}
\begin{itemize}
    \item Pokračovať v trénovaní modelov na verejných datasetoch.
    \item Testovať alternatívne formy predspracovania.
    \item Nové backendové a frontendové funkcionality sa budú ďalej stabilizovať a vyvíjať.
    \item Dataset anotovaných historických dokumentov sa rozšíri.
\end{itemize}

\vspace{0.5cm}
\textbf{Úlohy do nasledujúceho stretnutia:}
\begin{itemize}
    \item Bc. Michal Balogh: Pokračovať v trénovaní na verejných datasetoch, pripraviť finetuning a anotovať ďalšie historické dokumenty
    \item Bc. Juraj Hušek: Pokračovať v implementácii API a anotovať ďalšie historické dokumenty
    \item Bc. Ján Osadský: Pokračovať v implementácii API
    \item Bc. Fridrich Molnár: Pracovať na vylepšení UI/UX a anotovacieho nástroja
    \item Bc. Zoltán Renczes: Pracovať na vylepšení UI/UX a anotovacieho nástroja
\end{itemize}

\vspace{0.5cm}
\textbf{Nasledujúce stretnutie:} \\
Dátum: 20. novembra 2025 \\
Čas: 12:00 \\
Umiestnenie: Slovenská technická univerzita v Bratislave, Fakulta elektrotechniky a informatiky \\
Miestnosť: A208, FEI STU

\end{document}
