\subsubsection{Úvodné stretnutie tímu – 25. septembra 2025}

\textbf{Dátum:} 25. septembra 2025 \\
\textbf{Čas:} 12:00 – 13:00 \\
\textbf{Umiestnenie:} Slovenská technická univerzita v Bratislave, Fakulta elektrotechniky a informatiky \\
\textbf{Miestnosť:} A208, FEI STU \\
\textbf{Zapisovateľ:} Bc. Juraj Hušek

\vspace{0.5cm}
\textbf{Účastníci:}
\begin{itemize}
    \item Bc. Michal Balogh – ML/AI Engineer
    \item Bc. Juraj Hušek – Backend Developer
    \item Bc. Ján Osadský – Backend Developer
    \item Bc. Fridrich Molnár – Frontend Developer
    \item Bc. Zoltán Renczes – Frontend Developer
\end{itemize}

\textbf{Vedúci tímového projektu:} Ing. Stanislav Marochok

\vspace{0.5cm}
\textbf{Program stretnutia:}
\begin{enumerate}
    \item Oboznámenie sa s projektom
    \item Rozdelenie rolí v tíme
    \item Organizačné pokyny
    \item Diskusia o technológiách a funkcionalitách
    \item Plánovanie ďalších krokov
\end{enumerate}

\vspace{0.5cm}
\textbf{Diskusia a rozhodnutia:}

\paragraph{1. Oboznámenie sa s projektom} 
Vedúci projektu podrobne predstavil zadanie. Cieľom je vyvinúť plnohodnotnú webovú aplikáciu, ktorá umožní digitalizáciu, anotáciu, rozpoznávanie a dešifrovanie historických šifrovaných textov. Diskutovalo sa o výskumnom potenciáli aj praktickom prínose riešenia. Tím sa zhodol na potrebe modulárneho prístupu a postupného zavádzania funkcionalít.

\paragraph{2. Rozdelenie rolí} 
Tím sa dohodol na nasledovnom rozdelení úloh:
\begin{itemize}
    \item ML/AI Engineer: Bc. Michal Balogh
    \item Backend Developers: Bc. Juraj Hušek, Bc. Ján Osadský
    \item Frontend Developers: Bc. Fridrich Molnár, Bc. Zoltán Renczes
\end{itemize}
Rozdelenie vychádza z predchádzajúcich skúseností členov a zodpovedá potrebám projektu.

\paragraph{3. Organizačné pokyny} 
Stretnutia budú pravidelne každý štvrtok o 12:00 v miestnosti A208. GitHub bude využívaný ako spoločný repozitár zdrojového kódu webovej aplikácie, webovej stránky projektu a AI experimentov. Komunikácia ohľadom projektu bude prebiehať prezenčne a na platforme Discord. Tím sa zhodol, že je potrebné zaviesť pravidelné konzultácie s vedúcim tímového projektu, aby sa predišlo nejasnostiam a priebežne sa riešili otázky, problémy a možné ďalšie funkcionality.

\paragraph{4. Diskusia o technológiách a funkcionalitách} 
Diskutovalo sa o základnom návrhu architektúry a výbere technológií:
\begin{itemize}
    \item Backend: Python FastAPI alebo Flask
    \item Frontend: React
    \item Databáza: PostgreSQL alebo MySQL
    \item AI/ML: Python (OpenCV, TensorFlow alebo PyTorch)
    \item Deployment: Docker a Raspberry Pi 5 pre webovú aplikáciu, GitHub Pages pre informačnú stránku projektu
\end{itemize}

\textbf{Diskusia o úvodnej funkcionalite:}  
Diskutovali sme o návrhu základných krokov úvodnej funkcionality aplikácie, ktorá by mala umožniť používateľovi nahrať dokument do systému. Ide o pracovný návrh, ktorý nie je finálny a bude ešte spresňovaný v nasledujúcich konzultáciách:
\begin{enumerate}
    \item Nahratie dokumentu a klasifikácia – systém rozpozná, či ide o textový dokument. Ak dokument nebude textového charakteru, spracovanie sa zastaví.
    \item Preprocessing – plánované operácie: deskewing, binarizácia a odstránenie šumu, pričom používateľ bude mať možnosť nastavovať parametre.
    \item Detekcia obsahu – identifikácia tabuliek, textových oblastí alebo šifrovaných úsekov.
    \item Priradenie symbolov – pokus o mapovanie šifrovaných častí na znaky uložené v databáze a ponuka možných riešení používateľovi.
    \item Možnosť manuálnych zásahov – každý krok bude realizovateľný aj manuálne, neurónové siete budú slúžiť len ako asistent.
\end{enumerate}

\paragraph{5. Úlohy do nasledujúceho stretnutia}
\begin{itemize}
    \item Všetci členovia: detailne si preštudovať zadanie projektu.
    \item Bc. Michal Balogh: založiť GitHub projekt, vytvoriť príslušné repozitáre a pripraviť základnú verziu webovej stránky projektu.
    \item Bc. Juraj Hušek, Bc. Michal Balogh, Bc. Ján Osadský: vypracovať a navrhnúť use case diagramy webovej aplikácie.
    \item Bc. Fridrich Molnár a Bc. Zoltán Renczes: preskúmať možnosti a konfiguračné kroky potrebné na nasadenie aplikácie na Raspberry Pi server.
\end{itemize}

\vspace{0.5cm}
\textbf{Nasledujúce stretnutie:} \\
Dátum: 2. októbra 2025 \\
Čas: 12:00 \\
Umiestnenie: Slovenská technická univerzita v Bratislave, Fakulta elektrotechniky a informatiky \\
Miestnosť: A208, FEI STU

\vspace{0.5cm}
\textbf{Poznámky:}
\begin{itemize}
    \item Projekt má potenciál priniesť významné výsledky z pohľadu výskumu aj praktického využitia.
    \item Tím sa zhodol, že na jeho realizáciu je vhodné postupovať agilným spôsobom, aby bolo možné priebežne reagovať na nové požiadavky a zistenia.
    \item Zdôraznená bola potreba pravidelných konzultácií s vedúcim tímového projektu, ktoré umožnia včasnú identifikáciu a riešenie problémov a ďalších funkcionalít systému.
\end{itemize}
